  \section{Tirer partie de l'IaaS}

  \begin{frame}
    \frametitle{Deux visions}
    Une fois un cloud IaaS en place, deux optiques possibles :
    \begin{itemize}
      \item Garder les mêmes pratiques tout en profitant du self service et de l'agilité de la solution pour des besoins test/dev
      \item \textbf<2->{Faire évoluer ses pratiques}, tant côté applicatif que système
    \end{itemize}\pause
    "Pets vs Cattle"
  \end{frame}

  \begin{frame}
    \frametitle{Sinon ?}
    Faire tourner des applications \textit{legacy} dans le cloud est une mauvaise idée :
    \begin{itemize}
      \item Interruptions de service
      \item Pertes de données
      \item Incompréhensions "le cloud ça marche pas"
    \end{itemize}
  \end{frame}

  \subsection{Côté applications}

  \begin{frame}[allowframebreaks]
    \frametitle{Adapter ou penser ses applications "cloud ready"}
    Cf. les design tenets du projet OpenStack et Twelve-Factor \url{http://12factor.net/}
    \begin{itemize}
      \item Architecture distribuée plutôt que monolithique
      \begin{itemize}
        \item Facilite le passage à l'échelle
        \item Limite les domaines de \textit{failure}
      \end{itemize}\pause
      \item Couplage faible entre les composants
      \item Bus de messages pour les communications inter-composants\framebreak
      \item Stateless : permet de multiplier les routes d'accès à l'application\pause
      \item Dynamicité : l'application doit s'adapter à son environnement et se reconfigurer lorsque nécessaire\pause
      \item Permettre le déploiement et l'exploitation par des outils d'automatisation\pause
      \item Limiter autant que possible les dépendances à du matériel ou du logiciel spécifique qui pourrait ne pas fonctionner dans un cloud\pause
      \item Tolérance aux pannes (\textit{fault tolerance}) intégrée\pause
      \item Ne pas stocker les données en local, mais plutôt :
      \begin{itemize}
        \item Base de données
        \item Stockage objet
      \end{itemize}\pause
      \item Utiliser des outils standards de journalisation
    \end{itemize}
  \end{frame}

  \subsection{Côté système}

  \begin{frame}
    \frametitle{Adopter une philosophie DevOps}
    \begin{itemize}
      \item Infrastructure as Code
      \item Scale out plutôt que scale up (horizontalement plutôt que verticalement)
      \item HA niveau application plutôt qu'infrastructure
      \item Automatisation, automatisation, automatisation
      \item Tests
    \end{itemize}
  \end{frame}

  \begin{frame}
    \frametitle{Monitoring et backup}
    Monitoring
    \begin{itemize}
      \item Prendre en compte le cycle de vie des instances
      \item Monitorer le service plus que le serveur
    \end{itemize}
    Backuper, quoi ?
    \begin{itemize}
      \item Être capable de recréer ses instances (et le reste de son infrastructure)
      \item Données (applicatives, logs) : block, objet
    \end{itemize}
  \end{frame}

  \begin{frame}
    \frametitle{Utiliser des images cloud}
    Une image cloud c'est :
    \begin{itemize}
      \item Une image disque contenant un OS déjà installé
      \item Une image qui peut être instanciée en n machines sans erreur
      \item Un OS sachant parler à l'API de metadata du cloud (cloud-init)
    \end{itemize}
    Détails : \url{http://docs.openstack.org/image-guide/content/ch\_openstack\_images.html}\\
    La plupart des distributions fournissent aujourd'hui des images cloud.
  \end{frame}

  \begin{frame}
    \frametitle{Cirros}
    \begin{itemize}
      \item Cirros est l'image cloud par excellence
      \item OS minimaliste
      \item Contient cloud-init
    \end{itemize}
    \url{https://launchpad.net/cirros}
  \end{frame}

  \begin{frame}
    \frametitle{Cloud-init}
    \begin{itemize}
      \item Cloud-init est un moyen de tirer partie de l'API de metadata, et notamment des user data
      \item L'outil est intégré par défaut dans la plupart des images cloud
      \item À partir des user data, cloud-init effectue les opérations de personnalisation de l'instance
      \item cloud-config est un format possible de user data
    \end{itemize}
  \end{frame}

  \begin{frame}[containsverbatim]
    \frametitle{Exemple cloud-config}
\begin{verbatim}
#cloud-config
mounts:
 - [ xvdc, /var/www ]
packages:
 - apache2
 - htop
\end{verbatim}
  \end{frame}

  \begin{frame}
    \frametitle{Comment gérer ses images ?}
    \begin{itemize}
      \item Utilisation d'images génériques et personnalisation à l'instanciation
      \item Création d'images intermédiaires et/ou totalement personnalisées : \textit{Golden images}
      \begin{itemize}
        \item libguestfs, virt-builder, virt-sysprep
        \item diskimage-builder (TripleO)
        \item Packer
        \item solution "maison"
      \end{itemize}
    \end{itemize}
  \end{frame}

  \begin{frame}
    \frametitle{Configurer et orchestrer ses instances}
    \begin{itemize}
      \item Outils de gestion de configuration (les mêmes qui permettent de déployer OpenStack)
      \item Juju
    \end{itemize}
  \end{frame}
