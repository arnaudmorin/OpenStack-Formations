  \section{Tirer partie de l'IaaS}

  \begin{frame}
    \frametitle{Deux visions}
    Une fois un cloud IaaS en place, deux optiques possibles :
    \begin{itemize}
      \item Garder les mêmes pratiques tout en profitant du self service et de l'agilité de la solution
      \item Faire évoluer ses pratiquer, tant côté applicatif que système
    \end{itemize}
    "Pets vs Cattle"
  \end{frame}

  \subsection{Côté applications}

  \begin{frame}
    \frametitle{Adapter ou faire ses applications "cloud ready"}
    \begin{itemize}
      \item Stateless : permet de multiplier les routes d'accès à l'application
      \item Ne pas stocker les données en local, mais plutôt :
      \begin{itemize}
        \item Base de données
        \item Stockage objet
      \end{itemize}
    \item Utiliser des outils standards de journalisation
    \end{itemize}
  \end{frame}

  \subsection{Côté système}

  \begin{frame}
    \frametitle{Adopter une philosophie DevOps}
    \begin{itemize}
      \item Infrastructure as Code
      \item Scale out au lieu de scale up
      \item HA niveau application plutôt qu'infrastructure
    \end{itemize}
  \end{frame}

  \begin{frame}
    \frametitle{Utiliser des images cloud}
    Une image cloud c'est :
    \begin{itemize}
      \item Une image disque contenant un OS déjà installé
      \item Une image qui peut être instanciée en n machines sans erreur
      \item Un OS sachant parler à l'API de metadata du cloud (cloud-init)
    \end{itemize}
    La plupart des distributions fournissent aujourd'hui des images cloud.
  \end{frame}

  \begin{frame}
    \frametitle{Cirros}
    \begin{itemize}
      \item Cirros est l'image cloud par excellence
      \item OS minimaliste
      \item Contient cloud-init
      \item \url{https://launchpad.net/cirros}
    \end{itemize}
  \end{frame}

  \begin{frame}
    \frametitle{Faire ou modifier une image cloud}
    \begin{itemize}
      \item Utilisation de libguestfs
    \end{itemize}
  \end{frame}

  \begin{frame}
    \frametitle{Cloud-init}
    \begin{itemize}
      \item Cloud-init est un moyen de tirer partie de l'API de metadata, et notamment des user data
      \item L'outil est intégré par défaut dans la plupart des images cloud
      \item À partir des user data, cloud-init effectue les opérations de personnalisation de l'instance
      \item cloud-config est un format possible de user data
    \end{itemize}
  \end{frame}

  \begin{frame}[containsverbatim]
    \frametitle{Exemple cloud-config}
\begin{verbatim}
#cloud-config
mounts:
 - [ xvdc, /var/www ]
packages:
 - apache2
 - htop
\end{verbatim}
  \end{frame}

  \begin{frame}
    \frametitle{Configurer et orchestrer ses instances}
    \begin{itemize}
      \item Outils de gestion de configuration (les mêmes qui permettent de déployer OpenStack)
      \item Juju
    \end{itemize}
  \end{frame}
