  \section*{Introduction}
  \begin{frame}
    \frametitle{Objectifs de la formation}
    \begin{itemize}
      \item Virtualisation
      \item Cloud
      \item Focus sur OpenStack
    \end{itemize}
  \end{frame}
  
  \begin{frame}
    \frametitle{Objectifs de la formation : Virtualisation}
    \begin{itemize}
      \item Comprendre les principes de la virtualisation et son intérêt
      \item Connaitre le vocabulaire inhérent à la virtualisation
      \item Avoir une vue d'ensemble sur les solutions existantes de virtualisation
    \end{itemize}
  \end{frame}
  
  \begin{frame}
    \frametitle{Objectifs de la formation : Cloud}
    \begin{itemize}
      \item Comprendre les principes du cloud et son intérêt
      \item Connaitre le vocabulaire inhérent au cloud
      \item Avoir une vue d'ensemble sur les solutions existantes en cloud public et privé
      \item Posséder les clés pour tirer partie au mieux de l'IaaS
      \item Pouvoir déterminer ce qui est compatible avec la philosophie cloud ou pas
      \item Adapter ses méthodes d'administration système à un environnement cloud
    \end{itemize}
  \end{frame}

  \begin{frame}
    \frametitle{Objectifs de la formation : OpenStack}
    \begin{itemize}
      \item Connaitre le fonctionnement du projet OpenStack et ses possibilités
      \item Comprendre le fonctionnement de chacun des composants d'OpenStack
      \item Pouvoir faire les bons choix de configuration
      \item Savoir déployer manuellement un cloud OpenStack pour fournir de l'IaaS
      \item Connaitre les bonnes pratiques de déploiement d'OpenStack
      \item Être capable de déterminer l'origine d'une erreur dans OpenStack
      \item Savoir réagir face à un bug
    \end{itemize}
  \end{frame}

% Commenté
\begin{comment}
  \begin{frame}
    \frametitle{Pré-requis de la formation}
    \begin{itemize}
      \item Compétences d'administration système Linux tel qu'Ubuntu
      \begin{itemize}
        \item Gestion des paquets
        \item LVM et systèmes de fichiers
      \end{itemize}
      \item Notions de virtualisation, KVM et libvirt
      \item Peut servir :
      \begin{itemize}
        \item À l'aise dans un environnement Python
      \end{itemize}
    \end{itemize}
  \end{frame}
\end{comment}

  \begin{frame}
    \frametitle{Plan}
    \tableofcontents[hideallsubsections]
  \end{frame}
